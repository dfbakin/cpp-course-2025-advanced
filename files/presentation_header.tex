\PassOptionsToPackage{backref=page}{hyperref}
% %%% === ADDITIONAL PACKAGES
% \usepackage{animate}
\usepackage{caption}
\usepackage{subcaption}
\usepackage{tikz}
\usepackage{cancel}
\usepackage{booktabs}
\usepackage{stmaryrd}
\usepackage[export]{adjustbox}
\usepackage{fontawesome5}
\usepackage{algorithmic}
\usepackage{amsmath, amssymb}
\usepackage{yfonts}
\graphicspath{{../resources}}


% % \LinesNumbered
\newcommand{\theHalgorithm}{\arabic{algorithm}}
% \newcommand{\theHtable}{\thetable}
\usepackage[ruled,vlined]{algorithm2e}
\renewcommand{\algorithmicrequire}{\textbf{Input:}}
\renewcommand{\algorithmicensure}{\textbf{Output:}}
\newcommand{\vect}[1]{\boldsymbol{\mathbf{#1}}}

% %%% === CONDITIONAL PACKAGE
\usepackage{ifthen}

%%% === TEMPLATE
\usepackage{fontspec}
\usepackage{polyglossia}

% Set the main language to Russian
% \setmainlanguage{russian}

% Set Computer Modern Unicode (CMU) as the main font for both Latin and Cyrillic
\newfontfamily\cyrillicfont{CMU Serif}
\newfontfamily\cyrillicfontsf{CMU Sans Serif}
\newfontfamily\cyrillicfonttt{CMU Typewriter Text}

\setmainfont{CMU Serif}          % For regular (serif) text
\setsansfont{CMU Sans Serif}      % For sans-serif text
\setmonofont{CMU Typewriter Text} % For monospaced (typewriter-style) text
\DeclareSymbolFontAlphabet{\mathbb}{AMSb}
\setmathfont{CMU Sans Serif}

\setbeamerfont{title}{series=\bfseries}
\setbeamerfont{frametitle}{series=\bfseries, size=\fontsize{12}{14}}
\setbeamerfont{normal text}{series=\mdseries}

\setbeamersize
{
    text margin left=0.214cm,
    text margin right=0.214cm
}

% \usefonttheme[onlymath]{serif}
\setbeamertemplate{bibliography item}{\insertbiblabel}
\setbeamertemplate{itemize items}[circle] % For level-1 itemize
\setbeamertemplate{itemize subitem}[circle] % For level-2 (subitems)
\setbeamertemplate{itemize subsubitem}[circle] % For level-3 (subsubitems)


\usenavigationsymbolstemplate{}

% %%% === ADDITIONAL COMMANDS
\newcommand*{\Scale}[2][4]{\scalebox{#1}{$#2$}}%
\newcommand{\argmin}{\operatornamewithlimits{argmin}}
\newcommand{\argmax}{\operatornamewithlimits{argmax}}
\newcommand{\la}{\langle}
\newcommand{\ra}{\rangle}

% %%% === BACKGROUND IMAGE SETUP
% \AtBeginDocument{
%   \ifthenelse{\isundefined{\bgimage}}{
%     % No background image specified
%   }{
%     \setbeamertemplate{title page}{
%       \begin{tikzpicture}[remember picture,overlay]
%         \node[anchor=center, xshift=0pt, yshift=0pt] at (current page.center) {
%           \includegraphics[width=1.05\paperwidth, height=1.05\paperheight]{\bgimage}
%         };
%         \node[fill=white, fill opacity=0.7, text opacity=1, inner sep=10pt, rounded corners=10pt] at (current page.center) {
%           \begin{minipage}{0.5\paperwidth}
%             \centering
%             \usebeamerfont{title}\inserttitle\par
%             \vspace*{0.5cm}
%             \usebeamerfont{author}\insertauthor\par
%             \vspace*{0.5cm}
%             \usebeamerfont{institute}\insertinstitute\par
%           \end{minipage}
%         };
%       \end{tikzpicture}
%     }
%   }
% }
